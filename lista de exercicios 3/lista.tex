\documentclass[]{article}

\usepackage[brazil]{babel}
\usepackage[utf8]{inputenc}
\usepackage{hyperref}
\usepackage{graphicx}
\usepackage{multirow}
\usepackage{rotating}
\usepackage{latexsym}
\usepackage{subfigure}
\usepackage{amsmath}
\usepackage{amsfonts}
\usepackage{amssymb}
\usepackage{amsthm}
\usepackage{float}
\usepackage{fancyhdr}
%\usepackage{breakurl}
\usepackage{tabularx}
\usepackage{indentfirst}
\usepackage{mathtools}
\usepackage{pdfpages}
\usepackage{threeparttable}
\usepackage{adjustbox}

\usepackage{pgfplots}
\pgfplotsset{width=7cm,compat=1.8}

\usepackage{tikz}
\usetikzlibrary{positioning}

\usepackage{caption}

\usepackage[portuguese, ruled, linesnumbered]{algorithm2e}

\usepackage[margin=3cm]{geometry}

\usepackage{pdfpages}

%opening
\title{Lista de Exercícios - Projeto e Análise de Algoritmos}
\author{Luiz Alberto do Carmo Viana}

\begin{document}

\maketitle

\vspace{\baselineskip}

\textbf{Questão 1}

Dois grafos $G_1 = (V_1, E_1)$ e $G_2 = (V_2, E_2)$ são isomorfos se
existe uma função $f : V_1 \to V_2$ tal que $\{u, v\} \in E_1$ sse
$\{f(u), f(v)\} \in E_2$.  Em palavras, grafos isomorfos são
estruturalmente idênticos.  O problema \texttt{ISOMORPHIC}$(G_1, G_2)$
consiste em decidir se $G_1$ é isomorfo a algum subgrafo de $G_2$.
Mostre que \texttt{ISOMORPHIC} é NP-completo.  Dica: talvez seja
possível usar $G_1$ para buscar por certas estruturas dentro de $G_2$.

\vspace{\baselineskip}

\textbf{Questão 2}

Dados um subconjunto de naturais $S$ e um natural $t$,
\texttt{SUBSET-SUM}$(S, t)$ consiste em decidir se há um subconjunto
de $S$ cuja soma dos elementos é $t$.  Um outro problema
\texttt{SET-PARTITION}$(S)$ consiste em decidir se $S$ pode ser
particionado em dois subconjuntos de mesma soma.  Sabendo que
\texttt{SUBSET-SUM} é NP-completo, prove que \texttt{SET-PARTITION} é
NP-completo.

\vspace{\baselineskip}

\textbf{Questão 3}

O problema \texttt{HAMILTONIAN-PATH}$(G)$ consiste em decidir se $G$
possui um caminho hamiltoniano (que passa por todos os vértices).  Já
o problema \texttt{HAMILTONIAN-CYCLE}$(G)$ consiste em decidir se $G$
possui um ciclo hamiltoniano.  Sabendo que \texttt{HAMILTONIAN-PATH} é
NP-completo, prove que \texttt{HAMILTONIAN-CYCLE} é NP-completo.
Agora faça o contrário.

\vspace{\baselineskip}

\textbf{Questão 4}

O problema \texttt{HALF-SAT}$(\phi)$ consiste em decidir se há uma
valoração para as variáveis de uma fórmula CNF $\phi$ tal que
exatamente metade de suas $m$ cláusulas seja satisfeita.  Certamente,
\texttt{HALF-SAT}$(\phi)$ $= F$ se $m$ for ímpar, e portanto essas
instâncias não são interessantes.  Mostre que \texttt{HALF-SAT} é
NP-completo.

\vspace{\baselineskip}

\textbf{Questão 5}

O problema \texttt{HITTING-SET}$(\mathcal{S}, k)$ consiste em decidir
se, dada uma família de conjuntos
$\mathcal{S} = \{S_1, S_2, \dots, S_n\}$, existe um conjunto $H$ com
no máximo $k$ elementos tal que $H \cap S_i \neq \emptyset$, para
$1 \leq i \leq n$.  Mostre que \texttt{HITTING-SET} é NP-completo.

\vspace{\baselineskip}

\textbf{Questão 6}

O problema \texttt{MAX-DEG-SPANNING-TREE}$(G, k)$ consiste em decidir
se $G$ possui uma árvore geradora $T$ com $\Delta(T) \leq k$.  Prove
que esse problema é NP-completo.  Dica: você pode admitir que
\texttt{HAMILTONIAN-PATH} é NP-completo.

\vspace{\baselineskip}

\textbf{Questão 7}

Dado um grafo conexo $G = (V, E)$, um conjunto $S \subseteq V$ é dito
dominante se cada vértice de $G$ ou está em $S$ ou tem um vizinho em
$S$.  O problema \texttt{DOMINATING-SET}$(G, k)$ consiste em decidir
se $G$ possui um conjunto dominante com no máximo $k$ vértices.  Prove
que esse problema é NP-completo.  Dica: cobertura e dominância são
conceitos muito parecidos; toda cobertura de vértices é um conjunto
dominante, mas e a volta? Seria possível estender um grafo de forma
que a volta também fosse verdade?

\end{document}
%%% Local Variables:
%%% mode: latex
%%% TeX-master: t
%%% End:

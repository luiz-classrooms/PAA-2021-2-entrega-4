\documentclass[]{article}

\usepackage[brazil]{babel}
\usepackage[utf8]{inputenc}
\usepackage{hyperref}
\usepackage{graphicx}
\usepackage{multirow}
\usepackage{rotating}
\usepackage{latexsym}
\usepackage{subfigure}
\usepackage{amsmath}
\usepackage{amsfonts}
\usepackage{amssymb}
\usepackage{amsthm}
\usepackage{float}
\usepackage{fancyhdr}
%\usepackage{breakurl}
\usepackage{tabularx}
\usepackage{indentfirst}
\usepackage{mathtools}
\usepackage{pdfpages}
\usepackage{threeparttable}
\usepackage{adjustbox}

\usepackage{pgfplots}
\pgfplotsset{width=7cm,compat=1.8}

\usepackage{tikz}
\usetikzlibrary{positioning}

\usepackage{caption}

\usepackage[portuguese, ruled, linesnumbered]{algorithm2e}

\usepackage{pdfpages}

%opening
\title{Lista de Exercícios - Projeto e Análise de Algoritmos}
\author{Luiz Alberto do Carmo Viana}

\begin{document}

\maketitle

\vspace{\baselineskip}

\textbf{Questão 1}

Dado um grafo direcionado $G = (V, A)$, queremos determinar se $G$ possui uma celebridade, isto é, um vértice $v \in G$ que conhece nenhum vértice em $V \setminus v$, mas que é conhecido por todo vértice em $V \setminus v$.
Para determinar a existência de uma celebridade em $G$, podemos verificar, em tempo constante, se um par $(u, v) \in V^2$ pertence a $A$.
Desenvolva, por indução, um algoritmo que, dado um grafo direcionado $G = (V, A)$ como entrada, decide se $G$ possui uma celebridade.
Dica: dado um par $(u, v) \in V^2$ qualquer, podemos descartar ou $u$ ou $v$ da lista de candidatos a celebridade.
Isso nos ajuda a utilizar a hipótese da indução.

\vspace{\baselineskip}

\textbf{Questão 2}

Dados $a, b \in \mathbb{N}$, prove que $(n + a)^b \in \Theta(n^b)$.

\vspace{\baselineskip}

\textbf{Questão 3}

$2^{n + 1} \in O(2^n)$? $2^{2n} \in O(2^n)$?

\vspace{\baselineskip}

\textbf{Questão 4}

Dada uma função $g(n)$, prove que $o(g(n)) \cap \omega(g(n)) = \emptyset$.

\vspace{\baselineskip}

\textbf{Questão 5}

Prove que $n! \in \omega(2^n)$ e que $n! \in o(n^n)$.

\vspace{\baselineskip}

\textbf{Questão 6}

Dado $d \in \mathbb{N}$, tome o polinômio $p(n) = \sum_{i = 0}^d a_in^i$, onde $a_0, \dots, a_d \in \mathbb{R}$.
Para $p(n)$ ter grau $d$, vamos assumir que $a_d > 0$.
Prove que:
\begin{itemize}
  \item se $k \geq d$, então $p(n) \in O(n^k)$.
  \item se $k \leq d$, então $p(n) \in \Omega(n^k)$.
  \item se $k = d$, então $p(n) \in \Theta(n^k)$.
  \item se $k > d$, então $p(n) \in o(n^k)$.
  \item se $k < d$, então $p(n) \in \omega(n^k)$.
\end{itemize}

\vspace{\baselineskip}

\textbf{Questão 7}

Tome as funções \mbox{$(\frac{3}{2})^n, \ln (\ln n), 2^{\log n}$}, \mbox{$n^3, (\log n)^{\log n}, 2^{\sqrt{2 \log n}}, \sqrt{2}^{\log n}$}, \mbox{$\log^2 n, n 2^n, e^n$}, \mbox{$n, n^2, \log (n!), n^{\log (\log n)}$}, \mbox{$4^{\log n}, 2^n, n!$}, \mbox{$2^{2^n}, \ln n, (n + 1)!$}, \mbox{$n \log n, n^{\frac{1}{\log n}}, 1$}, \mbox{$\sqrt{\log n}, 2^{2^{n + 1}}$}.
Ordene-as assintoticamente.

\vspace{\baselineskip}

\textbf{Questão 8}

Tome $f(n)$ e $g(n)$ funções assintoticamente positivas.
Prove ou dê um contra-exemplo:
\begin{itemize}
  \item $f(n) \in O(g(n))$ implica que $g(n) \in O(f(n))$.
  \item $f(n) + g(n) \in \Theta(\min(f(n), g(n)))$.
  \item $f(n) \in O(g(n))$ implica que $\log (f(n)) \in O(\log (g(n)))$, desde que, para valores de $n$ suficientemente altos, $f(n) \geq 1$ e $\log (g(n)) \geq 1$.
  \item $f(n) \in O(g(n))$ implica que $2^{f(n)} \in O(2^{g(n)})$.
\end{itemize}

\vspace{\baselineskip}

\textbf{Questão 9}

Forneça um limite superior assintótico (em termos de $O$) para $T(n)$, onde::
\begin{itemize}
  \item
    \begin{align*}
      &T(1) = 1 \\
      &T(n) = T(n - 1) + 1
    \end{align*}
  \item
    \begin{align*}
      &T(1) = 1 \\
      &T(n) = T(n - 1) + n
    \end{align*}
  \item
    \begin{align*}
      &T(1) = 1 \\
      &T(n) = T(\frac{n}{2}) + 1
    \end{align*}
  \item
    \begin{align*}
      &T(1) = 1 \\
      &T(n) = 3T(\frac{n}{2}) + n
    \end{align*}
  \item
    \begin{align*}
      &T(1) = 1 \\
      &T(2) = 1 \\
      &T(n) = 4T(\frac{n}{3}) + n
    \end{align*}
  \item
    \begin{align*}
      &T(1) = 1 \\
      &T(n) = 4T(\frac{n}{2}) + n^2
    \end{align*}
  \item
    \begin{align*}
      &T(2) = 1 \\
      &T(n) = 3T(\sqrt{n}) + \log n
    \end{align*}
\end{itemize}

\vspace{\baselineskip}

\textbf{Questão 10}

Aplique o Teorema Mestre para:
\begin{itemize}
  \item $T(n) = 2T(\frac{n}{4}) + 1$
  \item $T(n) = 2T(\frac{n}{4}) + \sqrt{n}$
  \item $T(n) = 2T(\frac{n}{4}) + n$
  \item $T(n) = 2T(\frac{n}{4}) + n^2$
  \item $T(n) = T(\frac{n}{2}) + 1$
\end{itemize}

\vspace{\baselineskip}

\textbf{Questão 11}

O Teorema Mestre pode ser aplicado a $T(n) = 4T(\frac{n}{2}) + n^2\log n$?
De toda forma, forneça um limite superior assintótico para $T(n)$.

\vspace{\baselineskip}

\textbf{Questão 12}

Para resolver um certo problema, existem três algoritmos:
\begin{itemize}
  \item[1] resolve o problema dividindo-o em em cinco subproblemas com metade do tamanho da entrada, resolvendo-os recursivamente, e combinando suas soluções em tempo linear;
  \item[2] resolve o problema (com tanaho de entrada $n$) resolvendo recursivamente dois subproblemas de tamanho $n - 1$ e então combinando suas soluções em tempo constante;
  \item[3] resolve o problema dividindo-o em nove subproblemas com um terço do tamanho da entrada, resolvendo-os recursivamente, e combinando suas soluções em tempo quadrático.
\end{itemize}

Qual algoritmo deve ser escolhido para resolver esse problema?

\vspace{\baselineskip}

\textbf{Questão 13}

Dado um array de inteiros com $n$ posições, apresente um algoritmo que remova elementos duplicados em tempo $\Theta(n \log n)$.
Para fins de esclarecimento, se um array tem, digamos, duas posições com o elemento $3$, o array a ser produzido deve ter exatamente uma posição com o elemento $3$.

\vspace{\baselineskip}

\textbf{Questão 14}

Dado um array $a$ de inteiros com $n$ posições, apresente um algoritmo que determina se existe $1 \leq i \leq n$ tal que $a_i = i$.
Seu algoritmo deve ter complexidade $O(\log n)$.

\vspace{\baselineskip}

\textbf{Questão 15}

Tome um array infinito de inteiros em que as $n$ primeiras posições contém inteiros ordenados e as demais contém $\infty$.
O valor de $n$ não é conhecido a priori.
Descreva um algoritmo que, dado um inteiro $x$ como entrada, determina se o array contém $x$ em alguma de suas entradas.
O algoritmo deve ter custo $O(\log n)$.

\vspace{\baselineskip}

\textbf{Questão 16}

Dados $k$ arrays ordenados, cada um com $n$ inteiros, gostaríamos de combiná-los em um único array ordenado de $kn$ inteiros.
\begin{itemize}
  \item
    Usando o procedimento \texttt{merge} do algoritmo \texttt{mergesort}, combine os dois primeiros arrays, depois combine o array resultante com o terceiro, com o quarto, e assim por diante.
    Qual a complexidade dessa abordagem, em termos de $k$ e $n$?
  \item
    Utilizando divisão-e-conquista, desenvolva um algoritmo mais eficiente.
\end{itemize}

\vspace{\baselineskip}

\textbf{Questão 17}

Dados dois arrays ordenados com $m$ e $n$ inteiros, respectivamente, desenvolva um algoritmo $O(\log m + \log n)$ para encontrar o $k$-ésimo menor elemento na ``união'' dos dois arrays.

\vspace{\baselineskip}

\textbf{Questão 18}

Um array possui um elemento majoritário $e$ se mais da metade de suas posições são iguais a $e$.
Dado um array $a$ de $n$ posições, gostaríamos de determinar se $a$ possui um elemento majoritário e, em caso afirmativo, determinar esse elemento.
Não iremos assumir que nossos elementos são ordenáveis, mas apenas que são comparáveis por igualdade.
\begin{itemize}
  \item
    Mostre como resolver esse problema em tempo $O(n \log n)$
  \item
    Dados os $n$ elementos de $a$, crie $\frac{n}{2}$ pares.
    Para cada par: se ambos os elementos são iguais, guarde um deles; caso sejam diferentes, guarde nenhum deles.
    Mostre que você guardou no máximo $\frac{n}{2}$ elementos, e eles têm um elemento majoritário sse $a$ também tem.
  \item
    Qual a complexidade do algortimo descrito nessa abordagem?
\end{itemize}

\end{document}
%%% Local Variables:
%%% mode: latex
%%% TeX-master: t
%%% End:
